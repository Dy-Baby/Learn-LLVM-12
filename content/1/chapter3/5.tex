语义分析器遍历AST并检查语言的各种语义规则,例如:在使用变量之前必须声明变量,或者变量的类型必须在表达式中兼容。如果发现可以改进的情况,还可以输出警告。对于表达式语言,语义分析器必须检查每个使用的变量是否声明,这是语言所需。一个可能的扩展(这里不实现)是在未使用声明的变量时输出警告消息。\par

语义分析器在Sema类中实现,语义分析由semantic()执行。下面是完整的Sema.h头文件:\par

\begin{lstlisting}[caption={}]
#ifndef SEMA_H
#define SEMA_H

#include "AST.h"
#include "Lexer.h"

class Sema {
	public:
	bool semantic(AST *Tree);
};
#endif
\end{lstlisting}

实现在Sema.cpp文件中。有趣的部分是语义分析,并使用访问者实现。其基本思想是,每个声明变量的名称存储在一个集合中。创建时,可以检查每个命名是否唯一,然后检查命名是否在集合中:

\begin{lstlisting}[caption={}]
#include "Sema.h"
#include "llvm/ADT/StringSet.h"
namespace {
	class DeclCheck : public ASTVisitor {
		llvm::StringSet<> Scope;
		bool HasError;
		
		enum ErrorType { Twice, Not };
		void error(ErrorType ET, llvm::StringRef V) {
			llvm::errs() << "Variable " << V << " "
						 << (ET == Twice ? "already" : "not")
						 << " declared\n";
			HasError = true;
		}
	public:
	DeclCheck() : HasError(false) {}
	
	bool hasError() { return HasError; }
\end{lstlisting}

与Parser类中一样,使用标记来指示发生了错误,这些名字存储在Scope的集合中。在持有一个变量名的Factor节点中,我们检查该变量名是否在这个集合中。

\begin{lstlisting}[caption={}]
	virtual void visit(Factor &Node) override {
		if (Node.getKind() == Factor::Ident) {
			if (Scope.find(Node.getVal()) == Scope.end())
				error(Not, Node.getVal());
		}
	};
\end{lstlisting}

对于BinaryOp节点,只需要检查两边是否存在,并且是否已经访问过:

\begin{lstlisting}[caption={}]
	virtual void visit(BinaryOp &Node) override {
		if (Node.getLeft())
			Node.getLeft()->accept(*this);
		else
			HasError = true;
		if (Node.getRight())
			Node.getRight()->accept(*this);
		else
			HasError = true;
	};
\end{lstlisting}

在WithDecl节点中,填充集合并开始遍历表达式:

\begin{lstlisting}[caption={}]
	virtual void visit(WithDecl &Node) override {
		for (auto I = Node.begin(), E = Node.end(); I != E;
		++I) {
			if (!Scope.insert(*I).second)
				error(Twice, *I);
		}
		if (Node.getExpr())
			Node.getExpr()->accept(*this);
		else
			HasError = true;
		};
	};
}
\end{lstlisting}

semantic()方法会遍历树,并返回错误标志:

\begin{lstlisting}[caption={}]
bool Sema::semantic(AST *Tree) {
	if (!Tree)
		return false;
	DeclCheck Check;
	Tree->accept(Check);
	return Check.hasError();
}
\end{lstlisting}

如果没有使用声明的变量,也可以打印警告消息。这算是给读者留的课后作业。如果语义分析没有出现错误,那么就可以使用AST生成LLVM IR。我们将在下一节中完成这项工作。\par




























