我们在上一节构造的解析器只检查输入的语法,下一步是添加执行语义分析的能力。在前一章的calc例子中,解析器构造了AST。在另一个阶段,语义分析器会处理这棵树。在本节中,我们将使用一种不同的方法,并将解析器和语义分析器进一步融合在一起。\par

以下是语义分析器必须执行的一些任务:\par

\begin{itemize}
	\item 对于每个声明,语义分析器必须检查所使用的名称是否已经在其他地方声明过。
	\item 对于表达式或语句中每次出现的名称,语义分析程序必须检查名称是否已声明,以及期望的用途是否符合声明。
	\item 对于每个表达式,语义分析器必须计算得到的类型。还需要计算表达式是否为常量,如果是,它的值是多少。
	\item 对于赋值和参数传递,语义分析器必须检查类型是否兼容。此外,必须检查IF和WHILE语句中的条件是否为布尔类型。
\end{itemize}

对于这样一个编程语言的子集来说,要检查的内容已经很多了!\par

\hspace*{\fill} \par %插入空行
\textbf{处理名称的作用域}

让我们先看看名称的作用域,名称的范围是名称可见的范围。与C语言一样,tinylang也使用了先声明后使用的模型,例如:B和X变量是在模块级声明的,因此它们是INTEGER类型:\par

\begin{tcolorbox}[colback=white,colframe=black]
VAR B, X: INTEGER;
\end{tcolorbox}

在声明之前,变量是未知的,不能使用。这只有在声明之后才有可能。在过程中,可以声明更多的变量:\par

\begin{tcolorbox}[colback=white,colframe=black]
PROCEDURE Proc; \\
VAR B: BOOLEAN; \\
BEGIN \\
\hspace*{0.5cm}(* Statements *)\\
END Proc;
\end{tcolorbox}

在这个过程中,在注释所在的地方,使用B指的是局部变量B,而使用X指的是全局变量X。局部变量B的作用域是Proc过程。如果在当前作用域中找不到名称,则在外围作用域中继续搜索。因此,可以在过程中使用X变量。在tinylang中,只有模块和过程才能打开一个新的作用域。其他语言构造,如struct和class,通常也打开作用域。预定义(如INTEGER类型或TRUE字面值)在全局作用域中声明,该作用域中包含模块的作用域。\par

在tinylang,只有名字才是关键。因此,作用域可以实现为从名称到其声明的映射。只有在新名称不存在的情况下才能插入新名称。对于查找,封闭或父作用域也必须是已知的。该接口(在include/tinylang/Sema/Scope.h文件中)如下所示:\par

\begin{lstlisting}[caption={}]
#ifndef TINYLANG_SEMA_SCOPE_H
#define TINYLANG_SEMA_SCOPE_H

#include "tinylang/Basic/LLVM.h"
#include "llvm/ADT/StringMap.h"
#include "llvm/ADT/StringRef.h"

namespace tinylang {
	
class Decl;

class Scope {
	Scope *Parent;
	StringMap<Decl *> Symbols;
	
public:
	Scope(Scope *Parent = nullptr) : Parent(Parent) {}
	
	bool insert(Decl *Declaration);
	Decl *lookup(StringRef Name);
	
	Scope *getParent() { return Parent; }
};
} // namespace tinylang
#endif
\end{lstlisting}

lib/Sema/Scope.cpp文件中的实现如下所示:\par

\begin{lstlisting}[caption={}]
#include "tinylang/Sema/Scope.h"
#include "tinylang/AST/AST.h"

using namespace tinylang;

bool Scope::insert(Decl *Declaration) {
	return Symbols
		.insert(std::pair<StringRef, Decl *>(
			Declaration->getName(), Declaration))
		.second;
}
\end{lstlisting}

请注意,StringMap::insert()方法并不覆盖现有的规则条目。生成的std::pair的第二个成员表示表是否更新,该信息将返回给调用者。\par

为了实现对符号声明的搜索,lookup()方法搜索当前范围;如果没有找到,则搜索由父成员链接的作用域:\par

\begin{lstlisting}[caption={}]
Decl *Scope::lookup(StringRef Name) {
	Scope *S = this;
	while (S) {
		StringMap<Decl *>::const_iterator I =
			S->Symbols.find(Name);
		if (I != S->Symbols.end())
			return I->second;
		S = S->getParent();
	}
	return nullptr;
}
\end{lstlisting}

然后对变量声明进行如下处理:\par

\begin{itemize}
	\item 当前作用域是模块作用域。
	\item 查找INTEGER类型声明。如果没有找到任何声明,或者不是类型声明,则为错误。
	\item 实例化了新的AST节点VariableDeclaration,其中重要的属性是名称、B和类型。
	\item 将名称B插入到当前作用域,并映射到声明实例。如果名称已经出现在作用域中,那么是一个错误。这种情况下,当前范围的内容不会改变。
	\item 对X变量也做了同样的处理。
\end{itemize}

这里执行两个任务。与在calc示例中一样,构造了AST节点。同时,计算节点的属性,比如它的类型。为什么这样可行呢?\par

语义分析器可以依赖两组不同的属性。作用域是从调用者继承的,可以通过计算类型声明的名称来计算(或合成)类型声明。这种语言的设计方式使这两组属性足以计算AST节点的所有属性。\par

其中一个重要的方面是使用前声明模型。如果一种语言允许在声明之前使用名称,比如C++中的类中的成员,不可能一次计算AST节点的所有属性。这种情况下,AST节点必须仅使用部分计算的属性或纯信息构造(例如calc示例)。\par

必须访问AST一次或多次才能确定丢失的信息。在tinylang(和Modula-2)的情况下,也可以不使用AST构造——AST是通过parseXXX()方法的调用层次结构间接表示。从AST生成代码要常见得多,因此我们在这里构造了AST。\par

在我们把所有的部分放在一起之前,需要理解使用运行时类型信息(RTTI)的LLVM风格。\par

\hspace*{\fill} \par %插入空行
\textbf{在aAST中使用LLVM风格的RTTI}

当然,AST节点是类层次结构的一部分。声明总是有名字的,其他属性取决于声明的内容。如果声明了一个变量,则需要一个类型。常量声明需要类型和值等。当然,在运行时需要找出使用的是哪种声明。可以使用dynamic\underline{~}cast<> C++操作符。问题是,必需的RTTI只有在C++类附加了一个虚表时才可用;也就是说,使用虚函数。另一个缺点是C++ RTTI过于臃肿。为了避免这些缺点,LLVM开发人员引入了一种自制的RTTI风格。\par

层次结构的(抽象)基类是Decl。要实现LLVM风格的RTTI,需要添加一个包含每个子类标签的公共枚举。此外,还需要此类型的私有成员和公共getter,私有成员通常称为Kind。我们的例子中,它看起来像这样:\par

\begin{lstlisting}[caption={}]
class Decl {
public:
	enum DeclKind { DK_Module, DK_Const, DK_Type,
		DK_Var, DK_Param, DK_Proc };
private:
	const DeclKind Kind;
public:
	DeclKind getKind() const { return Kind; }
};
\end{lstlisting}

每个子类现在都需要一个名为classof的特殊函数成员。此函数的目的是确定给定实例是否属于请求的类型,对于VariableDeclaration,实现如下:\par

\begin{lstlisting}[caption={}]
static bool classof(const Decl *D) {
	return D->getKind() == DK_Var;
}
\end{lstlisting}

现在,可以使用llvm::isa<>特殊模板来检查对象是否为所请求的类型,并使用llvm::dyn\underline{~}cast<>来强制转换对象。还有更多的模板,但这两个是最常用的。其他模板请参阅\url{https://llvm.org/docs/ProgrammersManual.html\#the-isa-cast-and-dyn-cast-templates}。关于LLVM样式的更多信息,包括更高级的用法,请参阅\url{https://llvm.org/docs/HowToSetUpLLVMStyleRTTI.html}。\par

\hspace*{\fill} \par %插入空行
\textbf{创建语义分析器}

有了这些知识,我们现在可以实现语义分析器,在分析器创建的AST节点上操作。首先,我们将为一个变量实现AST节点的定义,该变量存储在include/llvm/tinylang/AST/AST.h文件中。除了支持LLVM风格的RTTI之外,基类还存储声明的名称、名称的位置和一个指向声明的指针,后者是代码生成嵌套过程所必需的。Decl基类的声明如下:\par

\begin{lstlisting}[caption={}]
class Decl {
	public:
		enum DeclKind { DK_Module, DK_Const, DK_Type,
						DK_Var, DK_Param, DK_Proc };
	
	private:
		const DeclKind Kind;
		
	protected:
		Decl *EnclosingDecL;
		SMLoc Loc;
		StringRef Name;
		
	public:
		Decl(DeclKind Kind, Decl *EnclosingDecL, SMLoc Loc,
			 StringRef Name)
			: Kind(Kind), EnclosingDecL(EnclosingDecL), Loc(Loc),
			  Name(Name) {}
	
	DeclKind getKind() const { return Kind; }
	SMLoc getLocation() { return Loc; }
	StringRef getName() { return Name; }
	Decl *getEnclosingDecl() { return EnclosingDecL; }
};
\end{lstlisting}

变量的声明只添加指向类型声明的指针:\par

\begin{lstlisting}[caption={}]
class TypeDeclaration;

class VariableDeclaration : public Decl {
	TypeDeclaration *Ty;
	
public:
	VariableDeclaration(Decl *EnclosingDecL, SMLoc Loc,
						StringRef Name, TypeDeclaration *Ty)
		: Decl(DK_Var, EnclosingDecL, Loc, Name), Ty(Ty) {}
	
	TypeDeclaration *getType() { return Ty; }
	
	static bool classof(const Decl *D) {
		return D->getKind() == DK_Var;
	}
};
\end{lstlisting}

解析器中的方法需要用语义动作和收集到的信息的变量进行扩展:\par

\begin{lstlisting}[caption={}]
bool Parser::parseVariableDeclaration(DeclList &Decls) {
	auto _errorhandler = [this] {
		while (!Tok.is(tok::semi)) {
			advance();
			if (Tok.is(tok::eof)) return true;
		}
		return false;
	};

	Decl *D = nullptr; IdentList Ids;
	if (parseIdentList(Ids)) return _errorhandler();
	if (consume(tok::colon)) return _errorhandler();
	if (parseQualident(D)) return _errorhandler();
	Actions.actOnVariableDeclaration(Decls, Ids, D);
	return false;
}
\end{lstlisting}

DeclList是一个名为std::vector<Decl*>的声明列表,而IdentList是一个类型为std::vector<std::pair<SMLoc, StringRef>>的位置和标识符列表。\par

parseQualident()方法返回一个声明。本例中,应该是一个类型声明。\par

解析器类知道语义分析器类Sema的一个实例,该实例存储在Actions成员中。对actOnVariableDeclaration()的调用运行语义分析器和AST构造。实现在lib/Sema/Sema.cpp文件中:\par

\begin{lstlisting}[caption={}]
void Sema::actOnVariableDeclaration(DeclList &Decls,
									IdentList &Ids,
									Decl *D) {
	if (TypeDeclaration *Ty = dyn_cast<TypeDeclaration>(D)) {
		for (auto I = Ids.begin(), E = Ids.end(); I != E; ++I) {
			SMLoc Loc = I->first;
			StringRef Name = I->second;
			VariableDeclaration *Decl = new VariableDeclaration(
				CurrentDecl, Loc, Name, Ty);
			if (CurrentScope->insert(Decl))
				Decls.push_back(Decl);
			else
				Diags.report(Loc, diag::err_symbold_declared, Name);
		}
	} else if (!Ids.empty()) {
		SMLoc Loc = Ids.front().first;
		Diags.report(Loc, diag::err_vardecl_requires_type);
	}
}
\end{lstlisting}

首先,使用llvm::dyn\underline{~}cast<TypeDeclaration>检查类型声明。如果不是类型声明,则打印错误消息。否则,对于Ids列表中的每个名称,将实例化一个VariableDeclaration并添加到声明列表中。如果将变量添加到当前作用域失败,因为名称已经声明,则打印一条错误消息。\par

大多数其他实体都是以相同方式构造,唯一不同的是它们语义分析的复杂性。模块和过程需要更多的工作,因为它们开辟了一个新的范围。开辟新的作用域很容易:只需要实例化一个新的scope对象。当模块或过程被解析,就必须删除作用域。\par

这必须以一种可靠的方式完成,因为我们不想在语法错误的情况下将名称添加到错误的作用域。这是C++中资源获取即初始化(RAII)习惯用法的经典用法。另一个复杂之处在于过程可以递归地调用自己。因此,在使用过程之前,必须将过程的名称添加到当前作用域,语义分析器有两种方法来进入和离开一个范围。作用域与声明相关联:\par

\begin{lstlisting}[caption={}]
void Sema::enterScope(Decl *D) {
	CurrentScope = new Scope(CurrentScope);
	CurrentDecl = D;
}

void Sema::leaveScope() {
	Scope *Parent = CurrentScope->getParent();
	delete CurrentScope;
	CurrentScope = Parent;
	CurrentDecl = CurrentDecl->getEnclosingDecl();
}
\end{lstlisting}

简单的helper类用于实现RAII:\par

\begin{lstlisting}[caption={}]
class EnterDeclScope {
	Sema &Semantics;
	public:
	EnterDeclScope(Sema &Semantics, Decl *D)
	: Semantics(Semantics) {
		Semantics.enterScope(D);
	}
~EnterDeclScope() { Semantics.leaveScope(); }
};
\end{lstlisting}

在解析模块或过程中,有两个与语义分析器的交互。第一个是在名称解析之后。这里,构造了(几乎为空的)AST节点,并建立了一个新的作用域:\par

\begin{lstlisting}[caption={}]
bool Parser::parseProcedureDeclaration(/* … */) {
	/* … */
	if (consume(tok::kw_PROCEDURE)) return _errorhandler();
	if (expect(tok::identifier)) return _errorhandler();
	ProcedureDeclaration *D =
		Actions.actOnProcedureDeclaration(
			Tok.getLocation(), Tok.getIdentifier());
	EnterDeclScope S(Actions, D);
	/* … */
}
\end{lstlisting}

语义分析器不只是检查当前作用域中的名称,还要返回AST节点:\par

\begin{lstlisting}[caption={}]
ProcedureDeclaration *
Sema::actOnProcedureDeclaration(SMLoc Loc, StringRef Name) {
	ProcedureDeclaration *P =
			new ProcedureDeclaration(CurrentDecl, Loc, Name);
	if (!CurrentScope->insert(P))
		Diags.report(Loc, diag::err_symbold_declared, Name);
	return P;
}
\end{lstlisting}

当解析了所有的声明和过程的主体,实际工作就完成了。基本上,语义分析器必须只检查过程声明末尾的名称是否等于过程的名称,以及用于返回类型的声明是否真的是类型声明:\par

\begin{lstlisting}[caption={}]
void Sema::actOnProcedureDeclaration(
	ProcedureDeclaration *ProcDecl, SMLoc Loc,
	StringRef Name, FormalParamList &Params, Decl *RetType,
	DeclList &Decls, StmtList &Stmts) {
		
	if (Name != ProcDecl->getName()) {
		Diags.report(Loc, diag::err_proc_identifier_not_equal);
		Diags.report(ProcDecl->getLocation(),
					 diag::note_proc_identifier_declaration);
	}
	ProcDecl->setDecls(Decls);
	ProcDecl->setStmts(Stmts);
	
	auto RetTypeDecl =
		dyn_cast_or_null<TypeDeclaration>(RetType);
	if (!RetTypeDecl && RetType)
		Diags.report(Loc, diag::err_returntype_must_be_type,
					 Name);
	else
		ProcDecl->setRetType(RetTypeDecl);
}
\end{lstlisting}

有些声明是固有的,不能由开发人员定义。这包括BOOLEAN和INTEGER类型以及TRUE和FALSE字面值。这些声明存在于全局作用域中,必须通过编程方式添加。Modula-2还预定义了一些,如INC或DEC,也应该添加到全局范围。给定我们的类,全局作用域的初始化很简单:\par

\begin{lstlisting}[caption={}]
void Sema::initialize() {
	CurrentScope = new Scope();
	CurrentDecl = nullptr;
	IntegerType =
		new TypeDeclaration(CurrentDecl, SMLoc(), "INTEGER");
	BooleanType =
		new TypeDeclaration(CurrentDecl, SMLoc(), "BOOLEAN");
	TrueLiteral = new BooleanLiteral(true, BooleanType);
	FalseLiteral = new BooleanLiteral(false, BooleanType);
	TrueConst = new ConstantDeclaration(CurrentDecl, SMLoc(),
										"TRUE", TrueLiteral);
	FalseConst = new ConstantDeclaration(
		CurrentDecl, SMLoc(), "FALSE", FalseLiteral);
	CurrentScope->insert(IntegerType);
	CurrentScope->insert(BooleanType);
	CurrentScope->insert(TrueConst);
	CurrentScope->insert(FalseConst);
}
\end{lstlisting}

有了这个方案,tinylang所需的所有计算都可以完成。例如,要计算表达式是否产生常量值,必须确保:\par

\begin{itemize}
	\item 常量声明的文字或引用是常量。
	\item 如果表达式的两边都是常量,应用该操作符也会得到一个常量。
\end{itemize}

在为表达式创建AST节点时,可以轻松地将这些规则嵌入到语义分析器中。同样,可以计算类型和常量值。\par

需要指出的是,并不是所有类型的计算都可以用这种方法来完成。例如,要检测未初始化变量的使用,可以使用一种称为符号解释的方法。在其一般形式中,该方法需要通过AST的特殊遍历顺序(这在构造期间是不可能的)。好消息是,本文介绍的方法创建了一个经过修饰的AST,它可以用于代码生成。当然,这个AST可以用于进一步的分析,可以根据需要打开或关闭分析。\par

要使用前端,还需要更新驱动程序。由于缺少代码生成,正确的tinylang程序不会产生输出。尽管如此,它仍可用于探索错误恢复和引发语义错误:\par

\begin{lstlisting}[caption={}]
#include "tinylang/Basic/Diagnostic.h"
#include "tinylang/Basic/Version.h"
#include "tinylang/Parser/Parser.h"
#include "llvm/Support/InitLLVM.h"
#include "llvm/Support/raw_ostream.h"

using namespace tinylang;

int main(int argc_, const char **argv_) {
	llvm::InitLLVM X(argc_, argv_);
	llvm::SmallVector<const char *, 256> argv(argv_ + 1,
											  argv_ + argc_);
	llvm::outs() << "Tinylang "
	 			 << tinylang::getTinylangVersion() << "\n";
	 			 
	for (const char *F : argv) {
		llvm::ErrorOr<std::unique_ptr<llvm::MemoryBuffer>>
			FileOrErr = llvm::MemoryBuffer::getFile(F);
		if (std::error_code BufferError =
				FileOrErr.getError()) {
			llvm::errs() << "Error reading " << F << ": "
						 << BufferError.message() << "\n";
			continue;
		}
	
		llvm::SourceMgr SrcMgr;
		DiagnosticsEngine Diags(SrcMgr);
		SrcMgr.AddNewSourceBuffer(std::move(*FileOrErr),
								  llvm::SMLoc());
		auto lexer = Lexer(SrcMgr, Diags);
		auto sema = Sema(Diags);
		auto parser = Parser(lexer, sema);
		parser.parse();
	}
}
\end{lstlisting}

恭喜你!已经完成了tinylang的前端实现!\par

现在,让我们尝试一下我们目前所学到的东西。以下是欧几里得最大公约数算法的实现的源代码,保存为Gcd.mod文件:\par

\begin{tcolorbox}[colback=white,colframe=black]
MODULE Gcd; \\
\\
PROCEDURE GCD(a, b: INTEGER):INTEGER; \\
VAR t: INTEGER; \\
BEGIN \\
\hspace*{0.5cm}IF b = 0 THEN RETURN a; END; \\
\hspace*{1cm}WHILE b \# 0 DO \\
\hspace*{1.5cm}t := a MOD b; \\
\hspace*{1.5cm}a := b; \\
\hspace*{1.5cm}b := t; \\
\hspace*{1cm}END; \\
\hspace*{1cm}RETURN a; \\
END GCD; \\
\\
END Gcd.
\end{tcolorbox}

让我们用下面的命令在这个文件上运行编译器:\par

\begin{tcolorbox}[colback=white,colframe=black]
\$ tinylang Gcm.mod \\
Tinylang 0.1
\end{tcolorbox}

除了打印的版本号,没有输出,这是因为只实现了前端部分。但是,如果您更改源代码,使其包含语法错误,则将打印错误消息。\par

下一章中,我们将通过添加代码生成来继续这项任务。\par

















