对于所有的应用程序,基本类型(如INTEGER)是不够用的。例如,要表示矩阵或复数等数学对象,必须基于现有数据类型构造新的数据类型。这些新的数据类型通常称为聚合或复合类型。\par

数组是同一类型元素的序列。在LLVM中,数组总是静态的:元素的数量是常量。整数数组[10]的tinylang类型,或长整数数组[10]的C类型,可以用IR表示为:\par

\begin{tcolorbox}[colback=white,colframe=black]
[10 x i64]
\end{tcolorbox}

结构是不同类型的复合。在编程语言中,通常用命名成员来表示。例如,在tinylang中,结构写为RECORD x, y: REAL; color: INTEGER; END;同样的结构在C中是struct \{float x, y;long color;\};。在LLVM IR中,只列出了类型名:\par

\begin{tcolorbox}[colback=white,colframe=black]
\{ float, float, i64 \}
\end{tcolorbox}

要访问成员,需要使用数字索引。与数组一样,第一个元素的索引号是0。\par

此结构的成员根据数据布局字符串中的规范在内存中进行布局。如果有必要,将插入未使用的填充字节。如果需要控制内存布局,那么可以使用打包结构,其中所有元素都是1字节对齐的。语法略有不同:\par

\begin{tcolorbox}[colback=white,colframe=black]
<{ float, float, i64 }>
\end{tcolorbox}

数组和结构被当作一个单元来处理。例如,不能将\%x数组的单个元素以\%x[3]引用。这时因为SSA的形式不能分辨\%x[i]和\%x[j]是否指相同的元素。相反,我们需要特殊的指令来提取和插入单个元素的值到数组中。要读取第二个元素,我们使用以下语句:\par

\begin{tcolorbox}[colback=white,colframe=black]
\%el2 = extractvalue [10 x i64] \%x, 1
\end{tcolorbox}

我们也可以更新一个元素,例如:第一个元素:\par

\begin{tcolorbox}[colback=white,colframe=black]
\%xnew = insertvalue [10 x i64] \%x, i64 \%el2, 0
\end{tcolorbox}

这两种指令也适用于结构体。例如,要从\%pt寄存器访问color成员,需要编写以下代码:\par

\begin{tcolorbox}[colback=white,colframe=black]
\%color = extractvalue { float, float, i64 } \%pt, 2
\end{tcolorbox}

这两个指令都有一个限制:索引必须是常数。对于结构来说,这很容易解释。索引号只是名称的替代,而C等语言没有动态计算结构成员名称的概念。对于数组,只是不能有效地实现。这两种指令在特定的情况下都有意义,当元素的数量很小且已知时,例如:复数可以建模为两个浮点数的数组。传递这个数组是合理的,并且在计算过程中访问数组的哪一部分是很清楚的。\par

对于前端的一般使用,需要求助于指向内存的指针。LLVM中的所有全局值都表示为指针,声明一个全局变量@arr,作为一个包含8个i64元素的数组,等价于长arr[8] C声明:\par

\begin{tcolorbox}[colback=white,colframe=black]
@arr = common global [8 x i64] zeroinitializer
\end{tcolorbox}

要访问数组的第二个元素,必须执行地址计算,以确定索引的地址。然后,可以从该地址加载该值。将其放入@second函数中,如下所示:\par

\begin{tcolorbox}[colback=white,colframe=black]
define i64 @second() { \\
\hspace*{0.5cm}\%1 = getelementptr [8 x i64], [8 x i64]* @arr, i64 0, i64 \\
\hspace*{0.5cm}1
\hspace*{0.5cm}\%2 = load i64, i64* \%1 \\
\hspace*{0.5cm}ret i64 \%2 \\
}
\end{tcolorbox}

getelementptr指令是地址计算的主要工具,所以需要对其进行更多的介绍。第一个操作数[8 x i64]是该指令操作的基类型。第二个操作数[8 x i64]* @arr指定基指针。注意这里的细微差别:我们声明了一个包含8个元素的数组,但所有全局值都视为指针,所以我们有一个指向数组的指针。C语法中,我们使用long (*arr)[8]!其结果是,在对元素进行索引之前,首先必须对指针进行解引用,例如:C中的arr[0][1]。第三个操作数i64 0对指针进行解引用,第四个操作数i64 1是元素的下标。计算的结果是索引元素的地址。请注意,本指令没有涉及内存。\par

除了结构之外,索引参数不需要是常量。因此,可以在循环中使用getelementptr指令来检索数组的元素。这里对结构的处理是不同的:只能使用常量,类型必须是i32。\par

有了这些知识,数组很容易集成到第5章的代码生成器中。必须扩展convertType()方法来创建类型,如果Arr变量持有数组的类型指示符,那么可以在方法中添加以下内容:\par

\begin{lstlisting}[caption={}]
llvm::Type *Component = convertType(Arr->getComponentType());
uint64_t NumElements = Arr->getNumElem();
return llvm::ArrayType::get(Component, NumElements);
\end{lstlisting}

该类型可用于声明全局变量。对于局部变量,需要为数组分配内存。我们在过程的第一个基本块中完成这个操作:\par

\begin{lstlisting}[caption={}]
for (auto *D : Proc->getDecls()) {
	if (auto *Var =
			llvm::dyn_cast<VariableDeclaration>(D)) {
		llvm::Type *Ty = mapType(Var);
		if (Ty->isAggregateType()) {
			llvm::Value *Val = Builder.CreateAlloca(Ty);
			Defs.Defs.insert(
				std::pair<Decl *, llvm::Value *>(Var, Val));
		}
	}
}
\end{lstlisting}

要读写一个元素,必须生成getelemtptr指令。这个指令会添加到emitExpr()(读取值)和emitAssign()(写入值)的方法中。要读取数组元素,首先读取变量的值,然后处理变量的选择器。对于每个索引,计算表达式并存储其值。基于这个列表,计算引用元素的地址并加载值:\par

\begin{lstlisting}[caption={}]
auto &Selectors = Var->getSelectorList();
for (auto *I = Selectors.begin(),
		*E = Selectors.end();
  	  I != E;) {
	if (auto *Idx = llvm::dyn_cast<IndexSelector>(*I)) {
		llvm::SmallVector<llvm::Value *, 4> IdxList;
		IdxList.push_back(emitExpr(Idx->getIndex()));
		for (++I; I != E;) {
			if (auto *Idx2 =
					llvm::dyn_cast<IndexSelector>(*I)) {
				IdxList.push_back(emitExpr(Idx2->getIndex()));
				++I;
			} else
			break;
		}
		Val = Builder.CreateGEP(Val, IdxList);
		Val = Builder.CreateLoad(
			Val->getType()->getPointerElementType(), Val);
	} else {
		llvm::report_fatal_error("Unsupported selector");
	}
}
\end{lstlisting}

写入数组元素使用相同的代码,但不生成加载指令。相反,在存储指令中使用指针作为目标。对于记录,可以使用类似的方法。记录成员的选择器包含常量字段索引,名为Idx。你可以通过以下方法将这个常量转换为一个LLVM常量值:\par

\begin{lstlisting}[caption={}]
llvm::Value *FieldIdx = llvm::ConstantInt::get(Int32Ty, Idx);
\end{lstlisting}

然后,您可以像使用数组一样使用Builder.CreateGEP()方法中的值。\par

现在您已经掌握了将聚合数据类型转换为LLVM IR的方法。以系统兼容的方式传递这些类型的值需要注意一些细节,在下一节中您将学习如何正确地实现它。\par













