静态分析器是一个令人印象深刻的例子,说明了使用Clang基础结构可以做什么。也可以通过插件来扩展Clang,这样你就可以将自己的功能添加到Clang中。该技术与在LLVM中添加Pass插件非常相似。\par

让我们用一个简单的插件来探索它的功能。LLVM编码标准要求函数名以小写字母开头。然而,随着时间的推移,编码标准也在不断发展,在许多情况下,函数以大写字母开头。一个产生违反命名规则的警告插件可以帮助解决这个问题,所以让我们尝试一下。\par

因为您希望在抽象语法树(AST)上运行用户定义的操作,所以需要定义PluginASTAction类的一个子类。如果你使用Clang库来编写自己的工具,那么可以将动作定义ASTFrontendAction类的子类。PluginASTAction类是ASTFrontendAction类的一个子类,具有解析命令行选项的额外功能。\par

需要的另一个类是ASTConsumer类的子类。AST消费者是一个类,你可以使用它在AST上运行操作,而不管AST的来源是什么。我们的第一个插件不需要太多组件。您可以在NamingPlugin.cpp文件中创建如下实现:\par

\begin{enumerate}
\item 首先包括必需的头文件。除了上面提到的ASTConsumer类,还需要一个编译器的实例和插件注册表:
\begin{lstlisting}[caption={}]
#include "clang/AST/ASTConsumer.h"
#include "clang/Frontend/CompilerInstance.h"
#include "clang/Frontend/FrontendPluginRegistry.h"
\end{lstlisting}

\item 使用clang命名空间,并将实现放到一个匿名命名空间中,以避免名称冲突:
\begin{lstlisting}[caption={}]
using namespace clang;
namespace {
\end{lstlisting}

\item 接下来,定义ASTConsumer类的子类。稍后,如果检测到违反命名规则,将发出警告。为此,您需要一个对DiagnosticsEngine实例的引用。

\item 需要在类中存储一个CompilerInstance实例,然后可以请求一个DiagnosticsEngine实例:
\begin{lstlisting}[caption={}]
class NamingASTConsumer : public ASTConsumer {
	CompilerInstance &CI;
	
public:
	NamingASTConsumer(CompilerInstance &CI) : CI(CI) {}
\end{lstlisting}

\item ASTConsumer实例有几个入口方法。HandleTopLevelDecl()方法最符合我们的要求,在顶层的每个声明都调用该方法。这不仅仅包括函数,例如:变量。因此,您将使用LLVM RTTI dyn\underline{~}cast<>()函数来确定该声明是否为函数声明。HandleTopLevelDecl()方法有一个声明组作为参数,它可以包含多个声明。这需要对声明进行循环。下面的代码向我们展示了HandleTop\allowbreak LevelDecl()方法:
\begin{lstlisting}[caption={}]
bool HandleTopLevelDecl(DeclGroupRef DG) override {
	for (DeclGroupRef::iterator I = DG.begin(),
								E = DG.end();
			I != E; ++I) {
		const Decl *D = *I;
		if (const FunctionDecl *FD =
			dyn_cast<FunctionDecl>(D)) {
\end{lstlisting}

\item 找到函数声明之后,您需要检索函数的名称。还需要确保名称不为空:
\begin{lstlisting}[caption={}]
			std::string Name =
				FD->getNameInfo().getName().getAsString();
			assert(Name.length() > 0 &&
				"Unexpected empty identifier");
\end{lstlisting}
如果函数名不是以小写字母开头,那么就违反了命名规则:
\begin{lstlisting}[caption={}]
			char &First = Name.at(0);
			if (!(First >= 'a' && First <= 'z')) {
\end{lstlisting}

\item 要发出警告,需要一个DiagnosticsEngine实例。此外,还需要一个消息ID。在Clang内部,消息ID定义为枚举。因为插件不是Clang的一部分,需要创建自定义ID,然后使用它发出警告:
\begin{lstlisting}[caption={}]
				DiagnosticsEngine &Diag =
					CI.getDiagnostics();
				unsigned ID = Diag.getCustomDiagID(
					DiagnosticsEngine::Warning,
					"Function name should start with "
					"lowercase letter");
				Diag.Report(FD->getLocation(), ID);
\end{lstlisting}

\item 除了关闭所有的开花括号外,还需要从这个函数返回true,表示处理可以继续:
\begin{lstlisting}[caption={}]
				}
			}
		}
	return true;
	}
};
\end{lstlisting}

\item 接下来,需要创建PluginASTAction子类,它实现了Clang调用的接口:
\begin{lstlisting}[caption={}]
class PluginNamingAction : public PluginASTAction {
public:
\end{lstlisting}
必须实现的第一个方法是CreateASTConsumer()方法,返回NamingASTConsumer类的一个实例。这个方法由Clang调用,传递的CompilerInstance实例可以访问编译器的所有重要类。下面的代码演示了这一点:
\begin{lstlisting}[caption={}]
	std::unique_ptr<ASTConsumer>
	CreateASTConsumer(CompilerInstance &CI,
					  StringRef file) override {
		return std::make_unique<NamingASTConsumer>(CI);
	}
\end{lstlisting}

\item 插件还可以访问命令行选项。您的插件没有命令行参数,只需要返回true来表示成功:
\begin{lstlisting}[caption={}]
	bool ParseArgs(const CompilerInstance &CI,
				   const std::vector<std::string> &args)
												 override {
		return true;
	}
\end{lstlisting}

\item 插件的操作类型描述了何时调用该操作。默认值是Cmdline,这意味着插件必须在命令行上命名才能调用。您需要重写该方法并将其值更改为AddAfterMainAction,插件才能自动运行:
\begin{lstlisting}[caption={}]
	PluginASTAction::ActionType getActionType() override {
		return AddAfterMainAction;
	}
\end{lstlisting}

\item PluginNamingAction类的实现完成了;只缺少类的结束大括号和匿名名称空间。将它们添加到代码中如下:
\begin{lstlisting}[caption={}]
};
}
\end{lstlisting}

\item 最后,需要注册插件。第一个参数是插件的名称,第二个参数是帮助文本:
\begin{lstlisting}[caption={}]
static FrontendPluginRegistry::Add<PluginNamingAction>
	X("naming-plugin", "naming plugin");
\end{lstlisting}

\end{enumerate}

这就完成了插件的实现。要编译插件,在CMakeLists.txt文件中创建一个构建描述。该插件位于Clang源代码树之外,因此您需要设置一个完整的项目。可以遵循以下步骤:\par

\begin{enumerate}
\item 首先定义所需的CMake版本和项目名称:
\begin{tcolorbox}[colback=white,colframe=black]
cmake\underline{~}minimum\underline{~}required(VERSION 3.13.4) \\
project(naminglugin)
\end{tcolorbox}

\item 接下来,包括LLVM文件。如果CMake不能自动找到文件,那么你必须设置LLVM\underline{~}DIR变量,以指向包含CMake文件的LLVM目录:
\begin{tcolorbox}[colback=white,colframe=black]
find\underline{~}package(LLVM REQUIRED CONFIG)
\end{tcolorbox}

\item 将带有CMake文件的LLVM目录添加到搜索路径中,并包含一些必需的模块:
\begin{tcolorbox}[colback=white,colframe=black]
list(APPEND CMAKE\underline{~}MODULE\underline{~}PATH \$\{LLVM\underline{~}DIR\}) \\ 
include(ChooseMSVCCRT) \\
include(AddLLVM) \\
include(HandleLLVMOptions)
\end{tcolorbox}

\item 然后,加载Clang的CMake定义。如果CMake不能自动找到文件,那么你必须设置Clang\underline{~}DIR变量指向包含CMake文件的Clang目录:
\begin{tcolorbox}[colback=white,colframe=black]
find\underline{~}package(Clang REQUIRED)
\end{tcolorbox}

\item 接下来,定义头文件和库文件的位置,以及使用哪些定义:
\begin{tcolorbox}[colback=white,colframe=black]
include\underline{~}directories("\$\{LLVM\underline{~}INCLUDE\underline{~}DIR\}" \\
\hspace*{1cm}"\$\{CLANG\underline{~}INCLUDE\underline{~}DIRS\}") \\
add\underline{~}definitions("\$\{LLVM\underline{~}DEFINITIONS\}") \\
link\underline{~}directories("\$\{LLVM\underline{~}LIBRARY\underline{~}DIR\}")
\end{tcolorbox}

\item 前面的定义设置了构建环境。插入以下命令,定义你的插件的名称,插件的源文件,以及它是一个Clang插件:
\begin{tcolorbox}[colback=white,colframe=black]
add\underline{~}llvm\underline{~}library(NamingPlugin MODULE NamingPlugin.cpp \\
\hspace*{3cm}PLUGIN\underline{~}TOOL clang)
\end{tcolorbox}

在Windows上,插件支持不同于Unix平台,必须链接所需的LLVM和Clang库:
\begin{tcolorbox}[colback=white,colframe=black]
if(LLVM\underline{~}ENABLE\underline{~}PLUGINS AND (WIN32 OR CYGWIN)) \\
\hspace*{0.5cm}set(LLVM\underline{~}LINK\underline{~}COMPONENTS Support) \\
\hspace*{0.5cm}clang\underline{~}target\underline{~}link\underline{~}libraries(NamingPlugin PRIVATE \\
\hspace*{1cm}clangAST clangBasic clangFrontend clangLex) \\
endif()
\end{tcolorbox}

\item 将这两个文件保存在NamingPlugin目录中。创建一个与NamingPlugin目录同级的build- NamingPlugin目录,并使用以下命令构建插件:
\begin{tcolorbox}[colback=white,colframe=black]
\$ mkdir build-naming-plugin \\
\$ cd build-naming-plugin \\
\$ cmake –G Ninja ../NamingPlugin \\
\$ ninja
\end{tcolorbox}

\end{enumerate}

这些步骤创建NamingPlugin。构建目录中的动态库。\par

要测试插件,请将下面的源代码保存为named.c文件。Func1函数名违反了命名规则,但主函数没有违反规则:\par

\begin{lstlisting}[caption={}]
int Func1() { return 0; }
int main() { return Func1(); }
\end{lstlisting}

要调用插件,需要使用-fplugin=指定:\par

\begin{tcolorbox}[colback=white,colframe=black]
\$ clang -fplugin=./NamingPlugin.so naming.c \\
naming.c:1:5: warning: Function name should start with \\
lowercase letter \\
int Func1() { return 0; } \\
\hspace*{0.7cm}\^ \\
1 warning generated.
\end{tcolorbox}

这种调用需要覆盖PluginASTAction类的getActionType()方法,并且返回一个不同于Cmdline默认值的值。\par

如果没有这样做,但还想对插件操作的调用有更多的控制,那么可以从编译器命令行运行插件:\par

\begin{tcolorbox}[colback=white,colframe=black]
\$ clang -cc1 -load ./NamingPlugin.so -plugin naming-plugin$\setminus$ \\
\hspace*{0.5cm}naming.c
\end{tcolorbox}

祝贺你,已经构建了你的第一个Clang插件!\par

这种方法的缺点是它有一定的局限性。ASTConsumer类有不同的入口方法,但它们都是粗粒度的。这可以通过使用RecursiveASTVisitor类来解决。这个类遍历所有AST节点,您可以覆盖感兴趣的VisitXXX()方法。你可以按照以下步骤来为使用“访问者”重写插件:\par

\begin{enumerate}
\item 定义RecursiveASTVisitor类需要一个额外的include:
\begin{lstlisting}[caption={}]
#include "clang/AST/RecursiveASTVisitor.h"
\end{lstlisting}

\item 然后,将访问者定义为匿名名称空间中的第一个类。您将只存储对AST上下文的引用,它将使您访问所有用于AST操作的重要方法,包括发出警告所需的DiagnosticsEngine实例:
\begin{lstlisting}[caption={}]
class NamingVisitor
	: public RecursiveASTVisitor<NamingVisitor> {
private:
	ASTContext &ASTCtx;
public:
	explicit NamingVisitor(CompilerInstance &CI)
		: ASTCtx(CI.getASTContext()) {}
\end{lstlisting}

\item 在遍历过程中,只要发现函数声明,就调用VisitFunctionDecl()方法。将内部循环的主体复制到HandleTopLevelDecl()函数中:
\begin{lstlisting}[caption={}]
	virtual bool VisitFunctionDecl(FunctionDecl *FD) {
		std::string Name =
			FD->getNameInfo().getName().getAsString();
		assert(Name.length() > 0 &&
			  "Unexpected empty identifier");
		char &First = Name.at(0);
		if (!(First >= 'a' && First <= 'z')) {
			DiagnosticsEngine &Diag =
				ASTCtx.getDiagnostics();
			unsigned ID = Diag.getCustomDiagID(
				DiagnosticsEngine::Warning,
				"Function name should start with "
				"lowercase letter");
			Diag.Report(FD->getLocation(), ID);
		}
		return true;
	}
};
\end{lstlisting}

\item 这就完成了访问者实现。在你的NamingASTConsumer类中,你现在只会存储一个访问者实例:
\begin{lstlisting}[caption={}]
	std::unique_ptr<NamingVisitor> Visitor;
public:
	NamingASTConsumer(CompilerInstance &CI)
		: Visitor(std::make_unique<NamingVisitor>(CI)) {}
\end{lstlisting}

\item 您将删除HandleTopLevelDecl()方法,因为该功能现在在访问者类中,所以您将需要重写HandleTranslationUnit()方法。每个翻译单元调用一次,会在这里开始AST遍历:
\begin{lstlisting}[caption={}]
	void
	HandleTranslationUnit(ASTContext &ASTCtx) override {
		Visitor->TraverseDecl(
			ASTCtx.getTranslationUnitDecl());
	}
\end{lstlisting}

\end{enumerate}

这个新的实现具有完全相同的功能。优点是它更容易扩展,例如:想检查变量声明,那么可以实现VisitVarDecl()方法。或者,如果想使用语句,那么可以实现VisitStmt()方法。基本上,对于C、C++和Objective-C语言的每个实体都有一个visitor方法。\par

访问AST允许构建执行复杂任务的插件。如本节所述,强制执行命名约定是对Clang的一个有用的补充。另一个可以作为插件实现的有用的附加功能是计算软件度量,比如圈复杂度。您还可以添加或替换AST节点,例如:允许添加运行时检测。添加插件可以以需要的方式,对Clang。进行扩展。\par









