准备好构建工具后,您现在可以从GitHub签出所有的LLVM项目。所有平台上执行此操作的命令本质上是相同的。但在Windows上,建议关闭对行结束符的自动转译。\par

我们分三部分来回顾这个过程:克隆存储库、创建构建目录和生成构建系统文件。\par

\hspace*{\fill} \par %插入空行
\textbf{克隆代码库}

在所有非windows平台上,输入以下命令克隆存储库:\par

\begin{tcolorbox}[colback=white,colframe=black]
	\$ git clone https://github.com/llvm/llvm-project.git
\end{tcolorbox}

在Windows上,您必须添加选项以禁用自动转译行结束符。在这里输入以下内容:\par

\begin{tcolorbox}[colback=white,colframe=black]
	\$ git clone --config core.autocrlf=false\ https://github.com/llvm/llvm-project.git
\end{tcolorbox}

这个git命令将最新的源代码从GitHub克隆到一个名为llvm-project的本地目录中。现在,用以下命令将当前目录更改为新的llvm-project目录:\par

\begin{tcolorbox}[colback=white,colframe=black]
	\$ cd llvm-project
\end{tcolorbox}

这个目录包含所有的LLVM项目,每个项目都在它自己的单独目录中。最值得注意的是,LLVM核心库位于LLVM子目录中。LLVM项目使用分支来进行后续版本开发(“release/12.x”)和标记(“llvmorg-12.0.0”)来标记某个版本。通过前面的clone命令,您可以获得当前的开发状态。本书使用LLVM 12。要查看LLVM 12的第一个版本,输入以下命令:\par

\begin{tcolorbox}[colback=white,colframe=black]
	\$ git checkout -b llvmorg-12.0.0
\end{tcolorbox}

这样,您就克隆了整个存储库并签出了一个标记。这是最灵活的方法。\par

Git还允许只克隆一个分支或标记(包括历史记录)。使用git clone—branch llvmorg-12.0.0 https://github.com/llvm/llvm-project。使用-depth=1选项,可以防止历史信息的克隆。这节省了时间和空间,但显然也限制了你能在本地可以做什么。\par

下一步就是创建一个构建目录。\par

\hspace*{\fill} \par %插入空行
\textbf{创建构建目录}

与许多其他项目不同,LLVM不支持内联构建,需要单独的\textit{build}目录。这可以很容易地在\textit{llvm-project}目录中创建。使用以下命令进入这个目录:\par

\begin{tcolorbox}[colback=white,colframe=black]
	\$ cd llvm-project
\end{tcolorbox}

然后,为了简单起见,创建一个名为build的构建目录。Unix和Windows系统的命令是不同,在类Unix系统上,应该使用以下命令:\par

\begin{tcolorbox}[colback=white,colframe=black]
	\$ mkdir build
\end{tcolorbox}

在Windows上,应该使用以下命令:\par

\begin{tcolorbox}[colback=white,colframe=black]
	\$ md build
\end{tcolorbox}

然后,切换到构建目录:\par

\begin{tcolorbox}[colback=white,colframe=black]
	\$ cd build
\end{tcolorbox}

现在,您可以在这个目录中使用CMake工具创建构建系统文件了。\par

\hspace*{\fill} \par %插入空行
\textbf{生成构建系统文件}

要生成将使用Ninja编译LLVM和Clang的构建系统文件,请运行以下命令:\par

\begin{tcolorbox}[colback=white,colframe=black]
	\$ cmake –G Ninja -DLLVM\underline{~}ENABLE\underline{~}PROJECTS=clang ../llvm
\end{tcolorbox}

\begin{tcolorbox}[colback=blue!5!white,colframe=blue!75!black,title=Tip]
	在Windows上,反斜杠字符$\setminus$是目录名分隔符,CMake会自动将Unix分隔符/转换为Windows分隔符。
\end{tcolorbox}

-G选项告诉CMake要为哪个系统生成构建文件。最常用的选项如下:\par

\begin{itemize}
	\item Ninja: 对应Ninja的构建系统
	\item Unix Makefiles: 对应GNU Make
	\item Visual Studio 15 VS2017和Visual Studio 16 VS2019: 对应Visual Studio和MS Build
	\item Xcode: 对应Xcode工程
\end{itemize}

可以使用-D选项设置各种变量来影响生成过程。通常,以CMAKE\underline{~}(由CMAKE定义)或LLVM\underline{~}(由LLVM定义)作为前缀。使用LLVM\underline{~}ENABLE\underline{~}PROJECTS=clang变量设置,CMake为LLVM之外的clang生成构建文件。命令的最后一部分告诉CMake在哪里可以找到LLVM核心库源代码。下一节中会有更多的相关内容。\par

当生成了构建文件,LLVM和Clang可以用以下命令编译:\par

\begin{tcolorbox}[colback=white,colframe=black]
	\$ ninja
\end{tcolorbox}

根据硬件资源的不同,该命令的运行时间在15分钟(具有大量CPU内核、内存和快速存储的服务器)到数小时(内存有限的双核Windows笔记本)之间。默认情况下,Ninja使用了所有可用的CPU核。这有利于提高编译速度,但可能会阻止其他任务的运行。例如,在Windows笔记本上,Ninja在运行时几乎不能上网。幸运的是,可以使用-j选项限制资源的使用。\par

让我们假设您有四个可用的CPU核,而Ninja应该只使用两个(因为您有并行任务要运行)。在这里,您应该使用以下命令进行编译:\par

\begin{tcolorbox}[colback=white,colframe=black]
	\$ ninja –j2
\end{tcolorbox}

当编译完成,可以运行测试套件,以检查是否一切正常:\par

\begin{tcolorbox}[colback=white,colframe=black]
	\$ ninja check-all
\end{tcolorbox}

同样,该命令的运行时因可用硬件资源的不同而有很大差异。Ninja检查目标运行所有测试用例。为每个包含测试用例的目录生成目标。使用check-llvm(而不是check-all)是运行LLVM测试,而不是Clang测试;check-llvm-codegen只运行来自LLVM的CodeGen目录中的测试(即llvm/test/CodeGen目录)。\par

也可以做一个快速的手动检查。您将使用的LLVM应用程序是llc,LLVM编译器。如果你使用-version选项,会显示它的LLVM版本,主机CPU,以及它所支持的所有架构:\par

\begin{tcolorbox}[colback=white,colframe=black]
	\$ bin/llc -version
\end{tcolorbox}

如果您在编译LLVM时有困难,那么可以参考LLVM系统文档入门中的常见问题部分(\url{https://llvm.org/docs/GettingStarted.html\#common-problems}),以获得常见问题的解决方案。\par

最后,安装二进制文件:\par

\begin{tcolorbox}[colback=white,colframe=black]
	\$ ninja install
\end{tcolorbox}

在类unix系统上,安装目录是/usr/local。在Windows下,使用C:$\setminus$Program Files$\setminus$LLVM。当然,这是可以修改的,下一节将说明如何操作。\par






