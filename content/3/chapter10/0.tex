The LLVM core libraries come with the ExecutionEngine component, which allows the compilation and execution of IR code in memory. Using this component, we can build just in time (JIT) compilers, which allow the direct execution of IR code. A JIT compiler works more like an interpreter, in the sense that no object code needs to be stored on secondary storage.\par

In this chapter, you will learn about applications for JIT compilers, and how the LLVM JIT compiler works in principle. You will explore the LLVM dynamic compiler and interpreter, and you will also learn how to implement a JIT compiler tool on your own. You will also see how to make use of a JIT compiler as part of a static compiler, and the challenges associated with it.\par

This chapter will cover the following topics:\par

\begin{itemize}
\item Getting an overview of LLVM's JIT implementation and use cases
\item Using JIT compilation for direct execution
\item Utilizing a JIT compiler for code evaluation
\end{itemize}

By the end of the chapter, you will know how to develop a JIT compiler, either using a preconfigured class, or a customized version fitting your needs. You will also acquire the knowledge to make use of a JIT compiler inside a traditional static compiler.\par










