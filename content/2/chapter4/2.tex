实际的编程语言比前一章简单的calc语言更复杂。为了更详细地研究它,我将在本章和接下来的章节中使用Modula-2的子集。Modula-2设计良好,可选地支持泛型和面向对象编程(OOP)(并不是说要在本书中创建一个完整的Modula-2编译器)。因此,我将把这个子集命名为tinylang。\par

让我们快速浏览一下tinylang语法的子集。在接下来的小节中,我们将从该语法派生词法分析器和解析器:\par

\begin{tcolorbox}[colback=white,colframe=black]
compilationUnit \\
\hspace*{0.5cm}: "MODULE" identifier ";" ( import )* block identifier "." ; \\
Import : ( "FROM" identifier )? "IMPORT" identList ";" ; \\
Block \\
\hspace*{0.5cm}: ( declaration )* ( "BEGIN" statementSequence )? "END" ;
\end{tcolorbox}

在Modula-2中,编译单元以MODULE关键字开头,然后是模块的名称。模块的内容可以是导入的模块列表、声明和在初始化时运行的语句的块:\par

\begin{tcolorbox}[colback=white,colframe=black]
declaration \\
\hspace*{0.5cm}: "CONST" ( constantDeclaration ";" )* \\
\hspace*{0.5cm}| "VAR" ( variableDeclaration ";" )* \\
\hspace*{0.5cm}| procedureDeclaration ";" ;
\end{tcolorbox}

声明引入常量、变量和过程。已声明的常量以CONST关键字作为前缀。类似地,变量声明以VAR关键字开头。声明常量非常简单:\par

\begin{tcolorbox}[colback=white,colframe=black]
constantDeclaration : identifier "=" expression ;
\end{tcolorbox}

标识符是常量的名称。该值派生自表达式,该表达式必须在编译时是可计算的。声明变量有点复杂:\par

\begin{tcolorbox}[colback=white,colframe=black]
variableDeclaration : identList ":" qualident ; \\
qualident : identifier ( "." identifier )* ; \\
identList : identifier ( "," identifier)* ;
\end{tcolorbox}

为了能够一次声明多个变量,必须使用标识符列表,类型的名称可能来自另一个模块。本例中以模块名作为前缀,称为限定标识符。而过程需要更多详细信息:\par

\begin{tcolorbox}[colback=white,colframe=black]
procedureDeclaration \\
\hspace*{0.5cm}: "PROCEDURE" identifier ( formalParameters )? ";" \\
\hspace*{1cm}block identifier ; \\
formalParameters \\
\hspace*{0.5cm}: "(" ( formalParameterList )? ")" ( ":" qualident )? ; \\
formalParameterList \\
\hspace*{0.5cm}: formalParameter (";" formalParameter )* ; \\
formalParameter : ( "VAR" )? identList ":" qualident ;
\end{tcolorbox}

在前面的代码中,您可以看到常量、变量和过程是如何声明的。过程可以有参数和返回类型。普通参数作为值传递,而VAR参数通过引用传递。前面的块规则缺少的另一部分是statementSequence,只是单个语句的列表:\par

\begin{tcolorbox}[colback=white,colframe=black]
statementSequence \\
\hspace*{0.5cm}: statement ( ";" statement )* ;
\end{tcolorbox}

如果语句后面跟着另一个语句,则该语句用分号分隔。同样,Modula-2语句支持这种方式:\par

\begin{tcolorbox}[colback=white,colframe=black]
statement \\
\hspace*{0.5cm}: qualident ( ":=" expression | ( "(" ( expList )? ")" )? ) \\
\hspace*{0.5cm}| ifStatement | whileStatement | "RETURN" ( expression )? ;
\end{tcolorbox}

该规则的第一部分描述了一个赋值或过程调用。后跟:=的限定标识符是一个赋值。另一方面,如果后面跟着(,则它是一个过程调用。其他语句是常用的控制语句:\par

\begin{tcolorbox}[colback=white,colframe=black]
ifStatement \\
\hspace*{0.5cm}: "IF" expression "THEN" statementSequence \\
\hspace*{0.5cm}( "ELSE" statementSequence )? "END" ;
\end{tcolorbox}

IF语句也有一个简化的语法,因为它只能有一个ELSE块。可以通过这个语句,我们有条件地保护语句:\par

\begin{tcolorbox}[colback=white,colframe=black]
whileStatement \\
\hspace*{0.5cm}: "WHILE" expression "DO" statementSequence "END" ;
\end{tcolorbox}

WHILE语句描述了一个由条件保护的循环。加上IF语句,就可以用tinylang编写简单的算法。最后,没有表达式的定义:\par

\begin{tcolorbox}[colback=white,colframe=black]
expList \\
\hspace*{0.5cm}: expression ( "," expression )* ; \\
expression \\
\hspace*{0.5cm}: simpleExpression ( relation simpleExpression )? ; \\
relation \\
\hspace*{0.5cm}: "=" | "\#" | "<" | "<=" | ">" | ">=" ; \\
simpleExpression \\
\hspace*{0.5cm}: ( "+" | "-" )? term ( addOperator term )* ; \\
addOperator \\
\hspace*{0.5cm}: "+" | "-" | "OR" ; \\
term \\
\hspace*{0.5cm}: factor ( mulOperator factor )* ; \\
mulOperator \\
\hspace*{0.5cm}: "*" | "/" | "DIV" | "MOD" | "AND" ; \\
factor \\
\hspace*{0.5cm}: integer\underline{~}literal | "(" expression ")" | "NOT" factor \\
| qualident ( "(" ( expList )? ")" )? ;
\end{tcolorbox}

表达式语法与前一章的calc相似,只支持INTEGER和BOOLEAN数据类型。\par

此外,还使用identifier和integer\underline{~}literal标记。\textbf{标识符}以字母或下划线开头,后面跟着字母、数字和下划线。\textbf{整数字面值}可以是一个十进制数字序列,也可以是一个十六进制数字序列,后面跟着字母H。\par

虽然已经介绍了很多规则,但只涉及Modula-2的一部分!尽管如此,还是可以使用这个子集中编写小型应用程序。接下来,让我们为tinylang实现一个编译器!\par



















