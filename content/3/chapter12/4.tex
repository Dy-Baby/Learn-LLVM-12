ELF文件格式是LLVM支持读写的二进制对象文件格式之一。ELF本身是为许多CPU架构结构定义的,对于M88k体系架构也有一个定义。我们需要做的就是添加重定位的定义和一些标志。重定位可在System V ABI M88k处理器手册的第4章看到:\par

\begin{enumerate}
\item 我们需要在llvm/include/llvm/BinaryFormat/ELFRelocs/M88k.def文件中输入以下内容:
\begin{lstlisting}[caption={}]
#ifndef ELF_RELOC
#error "ELF_RELOC must be defined"
#endif
ELF_RELOC(R_88K_NONE, 0)
ELF_RELOC(R_88K_COPY, 1)
// Many more…
\end{lstlisting}

\item 我们还在llvm/include/llvm/BinaryFormat/ELF.h文件中添加了一些标志,并包含了重定位定义:
\begin{lstlisting}[caption={}]
// M88k Specific e_flags
enum : unsigned {
	EF_88K_NABI = 0x80000000, // Not ABI compliant
	EF_88K_M88110 = 0x00000004 // File uses 88110-
	// specific
	// features
};
// M88k relocations.
enum {
#include "ELFRelocs/M88k.def"
};
\end{lstlisting}
代码可以添加到文件中的任何地方,但最好保持排序顺序,并在MIPS体系结构的代码之前插入它。

\item 我们还需要扩展其他一些方法。在llvm/include/llvm/Object/ELFObjectFile.h文件中,有一些方法可以在枚举成员和字符串之间进行转换,例如:我们必须在getFileFormatName()方法中添加新的case语句:
\begin{lstlisting}[caption={}]
	switch (EF.getHeader()->e_ident[ELF::EI_CLASS]) {
		// Many more cases
		case ELF::EM_88K:
			return "elf32-m88k";
	}
\end{lstlisting}

\item 类似地,我们扩展了getArch()方法。

\item 最后,我们在llvm/lib/Object/ELF.cpp文件的getELFRelocationTypeName()方法中使用重定位定义:
\begin{lstlisting}[caption={}]
	switch (Machine) {
		// Many more cases
		case ELF::EM_88K:
		switch (Type) {
#include "llvm/BinaryFormat/ELFRelocs/M88k.def"
			default:
			break;
		}
		break;
	}
\end{lstlisting}

\item 为了完成支持,还可以在llvm/lib/ObjectYAML/ELFYAML.cpp文件中,映射ELFYAML::\allowbreak ELF\underline{~}REL枚举的方法中添加重定位。

\item 至此,我们已经以ELF文件格式完成了对m88k体系结构的支持。您可以使用llvm-readobj工具检查ELF对象文件,例如:由OpenBSD上的交叉编译器创建的文件。同样,可以使用yaml2obj工具为m88k体系结构创建一个ELF对象文件。

\begin{tcolorbox}[colback=blue!5!white,colframe=blue!75!black, title=是否必须添加对对象文件格式的支持?]
	
将架构支持集成到ELF文件格式实现中只需要几行代码。如果您为其创建LLVM后端的体系结构使用ELF格式,那么您应该采用此方法。另一方面,添加对全新二进制文件格式的支持本身就是一项复杂的任务。在这种情况下,一种可能的方法是只输出汇编程序文件,并使用外部汇编程序创建对象文件。
\end{tcolorbox}

\end{enumerate}

通过这些添加,ELF文件格式的实现现在支持M88k体系结构。在下一节中,我们将创建M88k体系结构的描述,该描述介绍了该体系结构的指令、寄存器、调用规则,以及其他细节。\par























































