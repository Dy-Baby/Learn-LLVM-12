在上一节中,您看到许多Pass在后端运行。然而,这些Pass中的大多数不是在LLVM IR上运行,而是在MIR上运行。这是依赖于目标的指令表示,因此比LLVM IR更低层。它仍可以包含对虚拟寄存器的引用,因此它还不是纯CPU指令。\par

例如,要查看IR级别的优化,可以告诉llc在每个Pass完成后转储IR。这并不适用于后台的机器Pass,因为它们不在IR上工作。不过,MIR也有类似的功能。。\par

MIR是当前模块中机器指令当前状态的文本表示。它利用了YAML格式,允许序列化和反序列化。基本思想是,可以在某个点停止传递Pass,并以YAML格式检查状态。您还可以修改YAML文件,或者创建您自己的YAML文件,并传递它,并检查结果。这样可以方便地调试和测试。\par

让我们看一下MIR。使用\verb|--|stop-after=finalize-iseloption和我们之前使用的测试输入文件运行llc工具:\par

\begin{tcolorbox}[colback=white,colframe=black]
\$ llc -mtriple=mips-linux-gnu $\setminus$ \\
\hspace*{2cm}-stop-after=finalize-isel < sum.ll
\end{tcolorbox}

这指示llc在指令选择完成后转储MIR。缩短后的输出如下所示:\par

\begin{tcolorbox}[colback=white,colframe=black]
\verb|---| \\
name: \hspace{3cm}sum \\
body: \hspace{3cm}| \\
\hspace*{0.5cm}bb.0 (\%ir-block.0): \\
\\
\hspace*{0.5cm}liveins: \$a0, \$a1 \\
\hspace*{0.5cm}\%1:gpr32 = COPY \$a1 \\
\hspace*{0.5cm}\%0:gpr32 = COPY \$a0 \\
\hspace*{0.5cm}\%2:gpr32 = ADDu \%0, \%1 \\
\hspace*{0.5cm}\$v0 = COPY \%2 \\
\hspace*{0.5cm}RetRA implicit \$v0 \\
... 

\end{tcolorbox}

您可以注意几个属性。首先,虚拟寄存器(如\%0)和真实机器寄存器(如\$a0)的混合使用。原因在使用了更底层的ABI。为了跨不同的编译器和语言移植,函数遵循调用约定,这是应用程序二进制接口(ABI)的一部分。该输出用于MIPS机器上的Linux系统,需要使用系统使用的调用规则,第一个参数在寄存器\$a0中传递。MIR输出是在指令选择之后,但在寄存器分配之前生成的,所以仍然可以看到虚拟寄存器的使用。\par

MIR文件中使用的是机器指令ADDu,而不是LLVM IR中的add指令。您还可以看到虚拟寄存器附加了一个寄存器调用,本例中是gpr32。MIPS体系结构上没有16位寄存器,因此必须使用32位寄存器。\par

bb.0标签为第一个基本块,标签后的缩进内容是基本块的一部分。第一个语句指定在进入基本块时处于活动状态的寄存器。之后,在本例中,只有\$a0和\$a1这两个参数在输入时是活动的。\par

MIR文件中还有很多其他细节。你可以在LLVM MIR文档中进行了解 \url{https://llvm.org/docs/MIRLangRef.html}。\par

您遇到的第一个问题可能是如何找到一个Pass的名称,特别是当只需要检查该Pass之后的输出而不主动处理时。当使用带有llc的-debug-pass=Structure选项时,激活Pass的选项会打印在顶部。如果您想在Mips延迟槽填充器Paa之前停止,那么您需要查看打印的列表,并找到-mipsdelay-slot-filler选项,其也代表相应Pass的名称。\par

MIR文件格式的主要应用是在后端辅助测试机通过。使用llc和\verb|--|stop-after选项,在指定Pass后得到MIR。通常,您将使用它作为预期测试用例的基础。您要注意的第一件事是,MIR输出非常冗长。例如,许多字段是空的。为了减少这种混乱,可以在llc命令行中添加-simplify-mir选项。\par

您可以根据测试用例的需要保存和更改MIR。llc工具可以运行一个Pass,这是与MIR文件测试的完美匹配。我们假设您想要测试MIPS延迟槽填充器Pass。延迟槽是RISC体系结构(如MIPS或SPARC)的一个特殊属性:跳转后的下一条指令总是是执行。因此,编译器必须确保在每次跳转后都有合适的指令,而这个Pass就是执行这个任务的。\par

我们在运行Pass之前生成MIR:\par

\begin{tcolorbox}[colback=white,colframe=black]
\$ llc -mtriple=mips-linux-gnu $\setminus$ \\
\hspace*{2cm}-stop-before=mips-delay-slot-filler -simplify-mir $\setminus$ \\
\hspace*{2cm}< sum.ll >delay.mir
\end{tcolorbox}

输出要小很多,因为我们使用了-simplify-mir选项。函数体如下所示:\par

\begin{tcolorbox}[colback=white,colframe=black]
body: \hspace{3cm} | \\
\hspace*{0.5cm}bb.0 (\%ir-block.0): \\
\hspace*{1cm}liveins: \$a0, \$a1 \\
\\
\hspace*{1cm}renamable \$v0 = ADDu killed renamable \$a0, \\
\hspace*{6cm}killed renamable \$a1 \\
\hspace*{1cm}PseudoReturn undef \$ra, implicit \$v0
\end{tcolorbox}

最值得注意的是,您将看到ADDu指令后面是用于返回的apseudo指令。\par

delay.ll文件作为输入,现在运行延迟槽填充器Pass:\par

\begin{tcolorbox}[colback=white,colframe=black]
\$ llc -mtriple=mips-linux-gnu $\setminus$ \\
\hspace*{2cm}-run-pass=mips-delay-slot-filler -o - delay.mir
\end{tcolorbox}

现在将输出的函数与之前的函数进行比较:\par

\begin{tcolorbox}[colback=white,colframe=black]
body: \hspace{3cm} | \\
\hspace*{0.5cm}bb.0 (\%ir-block.0): \\
\hspace*{1cm}PseudoReturn undef \$ra, implicit \$v0 \{ \\
\hspace*{1.5cm}renamable \$v0 = ADDu killed renamable \$a0, \\
\hspace*{6cm}killed renamable \$a1
\end{tcolorbox}

您可以看到,用于返回的ADDu和伪指令已经更改了顺序,ADDu指令现在嵌套在返回语句中:传递标识了适合于延迟槽的ADDu指令。\par

如果您刚接触到延迟槽的概念,也会想要看看生成的组件,这可以通过llc很轻松的完成:\par

\begin{tcolorbox}[colback=white,colframe=black]
\$ llc -mtriple=mips-linux-gnu  < sum.ll
\end{tcolorbox}

输出包含了很多细节,但是在bb.0基础块的帮助下,可以很容易地找到为它生成的汇编代码:\par

\begin{tcolorbox}[colback=white,colframe=black]
\# \%bb.0: \\
\hspace*{2cm}jr\hspace{1cm} \$ra \\
\hspace*{2cm}jaddu\hspace{0.5cm} \$2, \$4, \$5
\end{tcolorbox}

的确,指令的顺序改变了!\par

有了这些知识,我们将了解后端的核心,并检查如何在LLVM中执行机器指令的选择。\par











